%---------change this every homework
\def\yourid{mst3k}
\def\collabs{list your collaborators here}
\def\sources{list your sources here}
% -----------------------------------------------------
\def\duedate{October 23, 2024 at 11:59p}
\def\pnumber{5}
%-------------------------------------

\documentclass[10pt]{article}
\usepackage{dsa2}
\usepackage{tikz-cd}


\begin{document}
\thispagestyle{empty}
\handout

%%%%%%%%%%%%%%%%%%%%%%%%%%%%%%%%%%%%%%%%%%%%%%%%%%%%%%%%
\begin{problem}Unit Intervals\end{problem}

You are given a set of points $P = \{p_1 , p_2 , ..., p_n \}$ on the real number line (you may assume these are given to you in sorted order). Create an algorithm that determines the smallest set of unit-length closed intervals that contains all of the given points. For example, the points $\{0.9, 1.2, 1.3, 2.1, 3.0\}$ can be covered by $[0.7, 1.7]$ and $[2.0, 3.0]$. The runtime of your algorithm should be no worse than $O(n)$. Describe your algorithm and explain it's runtime. Make an informal proof for your algorithm's correctness (always returning the optimal solution) using an exchange argument.  

\solution{
% Your solution here
}
%%%%%%%%%%%%%%%%%%%%%%%%%%%%%%%%%%%%%%%%%%%%%%%%%%%%%%%%

%%%%%%%%%%%%%%%%%%%%%%%%%%%%%%%%%%%%%%%%%%%%%%%%%%%%%%%%
\begin{problem} Crossing the Bridge \end{problem}
    $n$ people need to cross a narrow rope bridge as quickly as possible, and each respective person crosses at speeds $s_1, s_2, ... , s_n$ (you can assume these are integers and are sorted in descending order). You must also follow these additional constraints:

    \begin{enumerate}
        \item It is nighttime and you only have a single flashlight. One requires the flashlight to cross the bridge.
        \item A max of two people can cross the bridge together at one time (and they must have the flashlight).
        \item The flashlight must be walked back and forth, it cannot be thrown, mailed, raven'd, etc.
        \item A pair walking across together crosses at the speed of the slowest individual. They must stay together!
    \end{enumerate}

    Describe a greedy algorithm that solves this problem optimally and explain the runtime of your algorithm. \emph{NOTE: The obvious greedy algorithm does NOT work here. Be careful! This is more complicated than it appears.}


\solution{
% Your solution here
}


%%%%%%%%%%%%%%%%%%%%%%%%%%%%%%%%%%%%%%%%%%%%%%%%%%%%%%%%

\end{document}
